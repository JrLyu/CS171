\documentclass[12pt, a4paper]{article}

\usepackage[utf8]{inputenc}
\usepackage[framemethod=TikZ]{mdframed}
\usepackage{mathtools, amssymb, amsmath, cleveref, geometry, tcolorbox, graphicx, float, subfigure, arydshln, setspace, framed, pifont, physics, ntheorem, utopia}
%%% for coding %%%
\usepackage{listings}
\usepackage[ruled, vlined, linesnumbered]{algorithm2e}

\geometry{a4paper, left=1.2cm, right=1.2cm, bottom=1.5cm, top=1.2cm}

\definecolor{grey}{RGB}{177, 179, 179}
\definecolor{comment}{RGB}{96, 108, 120}
\definecolor{keyword}{RGB}{148, 57, 149}
\definecolor{string}{RGB}{180, 47, 36}

%%% for coding %%%
\lstset{language = java, basicstyle = \ttfamily\small,commentstyle = \rmfamily\color{comment}\textit, deletekeywords = {...}, escapeinside = {\%*}{*)}, frame = single, framesep = 0.5em, keywordstyle = \color{keyword}\textbf, morekeywords = {System, out}, emph = {self}, emphstyle=\bfseries\color{red}, numbers = left, numbersep = 1.5em, numberstyle = \ttfamily\small\color{grey},  rulecolor = \color{black}, showstringspaces = false, stringstyle = \ttfamily\color{string}, tabsize = 4, columns = flexible}

\newcounter{index}
\setcounter{index}{0}
\newenvironment*{df}[1]{\par\noindent\textbf{Definition \stepcounter{index}\theindex\ (#1).}}{\par}

\newenvironment*{eg}{\par\begin{tcolorbox}\noindent\textbf{Example}\quad}{\end{tcolorbox}\par}

\newenvironment*{thm}[1]{\begin{tcolorbox}\par\noindent\textbf{Theorem \stepcounter{index}\theindex\ #1} \par}{\par\end{tcolorbox}}

\newenvironment*{cor}[1]{\par\noindent\textbf{Corollary \thesubsection.\stepcounter{index}\theindex\ #1}}{\par}
\newenvironment*{lem}[1]{\par\noindent\textbf{Lemma \thesubsection.\stepcounter{index}\theindex\ #1}}{\par}
\newenvironment*{ax}[1]{\par\noindent\textbf{Axiom \thesubsection.\stepcounter{index}\theindex\ #1}}{\par}
\newenvironment*{prop}[1]{\par\noindent\textbf{Proposition \thesubsection.\stepcounter{index}\theindex\ #1}}{\par}
\newenvironment*{conj}[1]{\par\noindent\textbf{Conjecture \thesubsection.\stepcounter{index}\theindex\ #1}}{\par}
\newenvironment*{nota}{\par\noindent\textbf{Notation \thesubsection.\stepcounter{index}\theindex.}}{\par}

\newcounter{nprf}[subsection]
\setcounter{nprf}{0}
\newenvironment*{prf}{\par\indent\textbf{\textit{Proof \stepcounter{nprf}\thenprf.\quad}}}{\hfill$\blacksquare$\par}
\newenvironment*{dis}{\par\indent\textbf{\textit{Disproof \stepcounter{nprf}\thenprf.\quad}}}{\hfill$\blacksquare$\par}
\newenvironment*{sol}{\par \indent\textbf{\textit{Solution \stepcounter{nprf}\thenprf.\quad}}}{\hfill{$\square$}\par}

\newtheorem*{hint}{Hint.}
\newtheorem*{rmk}{Remark.}
\newtheorem*{ext}{Extension.}

\linespread{1.15}

\title{\textbf{This is a title}}
\author{Jiuru Lyu}
\date{\today}

\def\Z{{\mathbb{Z}}}
\def\R{{\mathbb{R}}}
\def\C{{\mathbb{C}}}
\def\Q{{\mathbb{Q}}}
\def\E{{\mathbb{E}}}
\def\d{{\mathrm{d}}}
\def\epsilon{\varepsilon}
\def\emptyset{\varnothing}
\def\O{\mathcal{O}}
\def\st{\emph{ s.t. }}

\begin{document}
\section*{Running Time/Complexity Analysis of Algorithms}
\subsection*{Intro to Algorithm Analysis}
\begin{itemize}
	\item We want to know how can we measure the ``goodness'' of a program/algorithm?
	\item We have some ways to measure:
	\begin{itemize}
		\item Running time (the shorter, the better)
		\item Memory utilization (the less the better)
		\item Amount of code (?)
		\item Etc.
	\end{itemize}
	\item The most commonly used performance measure is the \textbf{running time}.
	\item To measure running time, we can use a stop watch (i.e., use real time as measure). However, there are some problems associated: 
	\begin{itemize}
		\item The same program can have different running time on different computers
		\item Different inputs can result in different running time - hard to find their relationship
	\end{itemize}
	\item So, we need to rely on an more objective measure: count the number of instructions executed by a program for a given input size. 
	\item However, this measure is not practical. In practice, we will count the number of ``primitive operations'' executed by a program for a given input size. 
	\item Algorithms make repeated steps towards the solution. The \textbf{primitive operation} is a step in the algorithm. 
\begin{lstlisting}
	for (int i = 0; i < N; i++) {
		S1;
		S2;
		...
		SN
	}
\end{lstlisting}
The primitive operation of this algorithm consists of the statement \texttt{S1; S2; ...; SN}. 
	\item Principles of Algorithm Analysis
	\begin{itemize}
		\item Algorithm analysis consists of
		\begin{itemize}
			\item Determine frequency (=count) of primitive operations
			\item Characterize the frequency as a function of the input size
		\end{itemize}
		\item The algorithm analysis must
		\begin{itemize}
			\item Take into account all possible inputs (good ones and bad ones)
			\item Be independent of hardware/software environment
			\item Be independent from the programming language
			\item Give a good estimate that is proportional to the actual running time of the algorihtm
		\end{itemize}
	\end{itemize}
	\item Good inputs, Bad inputs, and Average cases
	\begin{itemize}
		\item Input data can affect the running time of algorithms
		\item The best case are not studied because we cannot count on luck.
		\item The worst case gives us an upper bound
		\begin{itemize}
			\item The worst case analysis provides an upper bound on the running time of an algorithm.
			\item The analysis is easier to do compare to average case analysis.
		\end{itemize}
		\item The average case is what we would expect.
		\begin{itemize}
			\item Take the average running time over all possible inputs of the same size
			\item The analysis depends on input distribution
			\item The analysis is harder to do because it uses probability techniques. 
		\end{itemize} 
	\end{itemize}
	\item Techniques used in Algorithm Analysis
	\begin{itemize}
		\item There are two main techniques used in Algorithm Analysis: 
		\begin{itemize}
			\item Loop analysis
			\item Recurrence relations
		\end{itemize}
		\item A program spends the most amount of time in loops. One of the technique used in algorithm analysis is loop analysis.
		\item Some algorithms are recursive. The running time complexity of recursive algorithms are often expressions as recurrence relations. Another technique is solving recurrence relations. \begin{eg}\[C(n)=2\times C(n/2)+1\]\end{eg}	

	\end{itemize}
\end{itemize}
\subsection*{Intro and the Big-Oh Notation}
\begin{itemize}
\item Consider the following program fragment, how many times is the loop body executed? 
\begin{lstlisting}
	double sum = 0
	for (int i = 0; i < n; i++) {
		sum += array[i];
	}
\end{lstlisting}
\begin{sol}$n$\end{sol}
\begin{lstlisting}
	double sum = 0
	for (int i = 0; i < n; i = i + 2) {
		sum += array[i];
	}
\end{lstlisting}
\begin{sol}$\frac{n}{2}$\end{sol}
\begin{lstlisting}
	double sum = 0
	for (int i = 0; i < n; i++) {
		for (int j = 0; j < n; ++) { 
			sum += array[i] * array[j]; 
		}
	}
\end{lstlisting}
\begin{sol}$n\times n=n^2$\end{sol}
\item The running time in terms of the input size ($n$) can be a general mathematical function. However, we are interested in the \textbf{order} of the growth function (but not the exact function). 
\item \begin{df}{Approximate Definition of Order, Similar/$\sim$}
 		Given 2 functions $f(x)$ and $g(x)$, we say $f(x)\sim g(x)$ if $\frac{f(x)}{g(x)}=1$ when $n\to\infty$. \end{df}
\begin{rmk}In running time analysis, we can ignore the less significant terms.\end{rmk}
\begin{df}{Precise Definition of Order, $\O$ notation}
Given two functions $f(n)$ and $g(n)$. The function $f(n)$ is $\O(g(n))$ (order of $g(n)$) if $\exists\ c,n_0\st f(n)\leq cg(n)\forall n\geq n_0$\end{df} \begin{rmk}A function $f(n)$ is Big-Oh of $g(n)$ if $f(n)\leq cg(n)$ for large values of $n$. That is, $f(n)$ is dominant by some multiple of $g(n)$ when $n$ is large. \end{rmk}
\begin{eg}
	$f(n)=2n+10$ is $\O(n)$. \begin{prf}For $n>10$, we have $2n+10<3n$. Therefore, we found $c=3$ and $n_0=10$ for which $f(n)\leq cg(n)$ when $n\geq n_0$.\end{prf}
	However, we know that $f(n)=n^2$ is not $\O(n)$. Picking $n=c+1$, the condition $n\leq c$ will never be satisfied. 
\end{eg}
\item Big-Oh and Growth rate
\begin{itemize}
	\item The Big-Oh notation gives an upper bound on the growth rate of a function $f(n)$ that represents the run time complexity of some algorithm
	\item If $f(n)$ is $\O(g(n))$, then the growth rate of $f(n)$ is no more than the growth rate of $g(n)$.
	\item In algorithm analysis, we use $\O(g(n))$ to rank (=categorize) functions by their growth rate. 
	\begin{eg}
		$2n+4$, $7n+9$, $10000n+999$ are all $\O(n)$, so in algorithm analysis we consider all these functions grow at the same rate. 	
	\end{eg}
\end{itemize}
\item Common Running Times: 
\begin{center}\begin{tabular}{c|c}
	$\O(1)$ & Constant Time\\
	$\O(\log(n))$ & Logarithmic\\ 
	$\O(n)$ & Linear\\
	$\O(n\log(n))$ & Log Linear\\
	$\O(n^2)$ & Quadratic
\end{tabular}\end{center}
\end{itemize}

\subsection*{Useful Formula in Algorithm Analysis}
\begin{itemize}
	\item Triangular Sums: \[1+2+3+4+\cdots+N=\dfrac{N(N+1)}{2}\approx\dfrac{N^2}{2}.\]
	\item Geometric Sums: \[1+2+4+8+\cdots+N(=2^n)=2N-1\approx 2N\]\[1+\dfrac{1}{2}+\dfrac{1}{4}+\cdots+\dfrac{1}{N}(=\dfrac{1}{2^n})=2-\dfrac{1}{N}\approx2\]
	\item Harmonic Sum: \[1+\dfrac{1}{2}+\dfrac{1}{3}+\cdots+\dfrac{1}{N}\approx\ln(N)\]
	\item Sterling's Approximation: \[\log(1)+\log(2)+\log(3)+\cdots+\log(N)=\log(N!)\approx N\log(N)\]
\end{itemize}

\subsection*{Loop Analysis}
\begin{lstlisting}
	for (int i = 0; i < 10; i++) {
		doPrimitive();
	}
\end{lstlisting}
\begin{itemize}
	\item The loop is executed $10$ times for any input size.
	\item Running time $=10$ \texttt{doPrimitive()} operations.
	\item Run time complexity $=\O(1)\implies$ constant time.
\end{itemize}
\begin{lstlisting}
	n = input size;  // (e.g.: # elements in an array)
	for (int i = 0; i < n; i++) {
		doPrimitive();
	}
\end{lstlisting}
\begin{itemize}
	\item The loop is executed $n$ times for an input size of $n$.
	\item Running time $=n$ \texttt{doPrimitive()} operations.
	\item Run time complexity $=\O(n)\implies$ linear
\end{itemize}
\begin{rmk}	$n$ or $N$ will always denote the input size in algorithm analysis. \end{rmk}
\begin{lstlisting}
	int sum = 0;
	for (int i = 0; i < 5*n; i = i + 4) {
		sum = sum + 1;
	}
\end{lstlisting}
\begin{itemize}
	\item The loop is executed $\frac{5}{4}n$ times for an input size of $n$.
	\item Running time $=\frac{5}{4}n$ \texttt{doPrimitive()} operations.
	\item Run time complexity $=\O(n)\implies$ linear
\end{itemize}
\begin{lstlisting}
	int sum = 0;
	for (int i = n; i > 0; i = i - 4) {
		sum = sum + 1;
	}
\end{lstlisting}
\begin{itemize}
	\item The loop is executed $\frac{1}{4}n$ times for an input size of $n$.
	\item Running time $=\frac{1}{4}n$ \texttt{doPrimitive()} operations.
	\item Run time complexity $=\O(n)\implies$ linear
\end{itemize}
\begin{lstlisting}
	int sum = 0;
	for (int i = 1; i <= n; i = 2*i) {
		sum++;
	}
\end{lstlisting}
\begin{itemize}
		\item $i$ will take the following numbers in the loop \[1\quad2\quad4\quad8\quad16\quad32\quad\cdots\]
		\item Loop will exists when $i>n$: \[1\quad2\quad4\quad8\quad16\quad32\quad\cdots\quad n\]
		\item Suppose $2^{k-1}\leq n\leq 2^k$ \[1\quad2\quad4\quad8\quad16\quad32\quad\cdots\quad2^{k-1}\quad n\quad2^{k}\]
		\item Iterations: $k\approx\log{n}$. So, $\O(\log{n})$. ($n\approx2^{k}\Longleftrightarrow k\approx\log(n)$)
\end{itemize}
\begin{lstlisting}
	int sum = 0;
	for (int i = n; i >= 1; i = i/2) {
		sum++;
	}
\end{lstlisting}
\begin{itemize}
	\item $i$ will take the following numbers in the loop
	\[n\quad n/2\quad n/4\quad\cdots\quad1\]
	\item Loop will exists when $i<1$.
	\item Suppose $n/2^{k}<1<n/2^{k-1}$. \[n\quad n/2\quad n/4\quad\cdots\quad n/2^{k-1}\quad1\quad n/2^{k}\]
	\item Iterations $k\approx\log{n}$. So, $\O(\log{n})$ ($n/2^k\approx1\implies n\approx2^k\implies k\approx\log{n}$)
\end{itemize}
\begin{lstlisting}
	int sum = 0;
	for (int i = 0; i < n; i++) {
		for (int j = 0; j <= i; j++) {
			sum++;
		}
	}
\end{lstlisting}
\begin{center}\begin{tabular}{c|c|c|c|c|c|c}
	$i$&0&2&3&4&$\cdots$&$n-1$\\\hline
	$j$&*&*&*&*&$\cdots$&*\\
	&&*&*&*&$\cdots$&*\\
	&&&*&*&$\cdots$&*\\
	&&&&*&$\cdots$&*\\
	&&&&&$\vdots$&*\\
	&&&&&&*
\end{tabular}\end{center}
\begin{itemize}
	\item We sum up those starts. That is adding from $1$ to $n$.
	\item Iterations $=\dfrac{n(n+1)}{2}$. So, $\O(n^2)$.
\end{itemize}
\begin{lstlisting}
	int sum = 0;
	for (int i = n; i > 0; i = i/2) {
		for (int j = 0; j < i; j++) {
			sum++;
		}
	}
\end{lstlisting}
\begin{center}\begin{tabular}{c|c|c|c|c|c|c}
	$i$&n&n/2&n/4&n/8&$\cdots$&1\\\hline
	$j$&*&*&*&*&$\cdots$&*\\
	&*&*&*&*&$\cdots$&\\
	&*&*&*&*&&\\
	&*&*&*&&&\\
	&*&*&&&&\\
	&*&&&&
\end{tabular}\end{center}
\begin{itemize}
	\item In total, we have $\log{n}$ $i$'s. We add up those starts. That is, $n+n/2+n/4+n/8+\cdots+1$.
	\item By Geometric sum, we have Iteration = $n(1+1/2+1/4+\cdots+1/n)\approx n(2)=2n$. 
	\item So, $\O(n)$.
\end{itemize}
\begin{lstlisting}
	int sum = 0;
	for (int i = 1; i <= n; i++) {
		for (int j = 0; j < n; j = j + i) {
			sum++;
		}
	}
\end{lstlisting}
\begin{center}\begin{tabular}{c|c|c|c|c|c|c}
	$i$&1&2&3&4&$\cdots$&n\\\hline
	$j$&0&0&0&0&$\cdots$&0\\
	&1&2&3&4&$\cdots$&\\
	&2&4&6&8&&\\
	&3&$\vdots$&$\vdots$&$\vdots$&&\\
	&4&$n-1$&$n-1$&$n-1$&&\\
	&$\vdots$&&&&\\
	&$n-1$&&&&
\end{tabular}\end{center}
\begin{itemize}
	\item When $i=1$, we iterate $j$ for $n$ times. When $i=2$, we iterate $j$ for $n/2$ times. For an arbitrary $i$, we iterate $j$ for $n/i$ times.
	\item So, iteration = $n+n/2+n/3+\cdots+n/n=n(1+1/2+1/3+\cdots+1/n)\approx n\log(n)$ by the harmonic series. So, $\O(n\log{n})$.
\end{itemize}

\subsection*{Analysis of Recursive Algorithms}
\begin{lstlisting}
	public static void recurse(int n) {
		if (n == 0) {
			doPrimitive();
		} else {
			doPrimitive();
			recurse(n-1);
		}
	}
\end{lstlisting}
\begin{itemize}
	\item Let $C(n)=$ \# of times that \texttt{doPrimitive()} is executed when input $=n$.
	\item $C(0)=1$ because when $n=0$, we only execute \texttt{doPrimitive()} one time and terminate. This is the base case. 
	\item $C(n)=1+C(n-1)$ for $n>0$. This is because \texttt{recurse(n)} will invoke \texttt{recurse(n-1)}, and by definition, the \# times that \texttt{doPrimitive()} is executed when input $=n-1$ is: $C(n-1)$
	\item To solve the recursive relation, we will use the technique \textbf{telescoping}.
	\begin{align*}
		C(n)&=1+C(n-1)\\
		&=1+1+C(n-2)\\
		&=1+1+1+C(n-3)\\
		&=\underbrace{1+1+1+\cdots+1}_{n\text {times}}+C(0)\\
		&=n+1
	\end{align*}
	\item So, $\O(n)$

\end{itemize}
\begin{lstlisting}
	public static void recurse(int n) {
		if (n == 0) {
			doPrimitive();
		} else {
			for (int i = 0; i < n; i++) {
				doPrimitive();
			}
			recurse(n-1);
		}
	}
\end{lstlisting}
\begin{itemize}
	\item Let $C(n)=$ \# of times that \texttt{doPrimitive()} is executed when input $=n$.
	\item $C(0)=1$ as the base case; $C(n)=n+C(n-1)$ for $n>0$ because \texttt{recurse(n)} will invoke \texttt{recurse(n-1)}, and by definition, the \# times that \texttt{doPrimitive()} is executed when input = $n-1$ is: $C(n-1)$.
	\item Telescoping: \begin{align*}
		C(n)&=n+C(n-1)\\
		&=n+(n-1)+C(n-2)\\
		&=n+(n-1)+(n-2)+C(n-3)\\
		&=n+(n-1)+(n-2)+\cdots+1+C(0)\\
		&=\dfrac{n(n+1)}{2}+1
 	\end{align*}
	\item So, $\O(n^2)$.
\end{itemize}
\begin{lstlisting}
	public static void recurse(int[] A, int a, int b) {
		if (b-a <= 1) {
			doPrimitive();
		} else {
			doPrimitive();
			recurse(A, a, (a+b)/2); // First half of array
			recurse(A, (a+b)/2, b); // 2nd half of array
		}
	}
\end{lstlisting}
\begin{itemize}
	\item Let $C(n)=$ \# of times that \texttt{doPrimitive()} is executed when input $=n$.
	\item $C(1)=1$ because if $b-a\leq1$, it executes $1$ \texttt{doPrimitive()}.
	\item  $C(n) = 1 + C(n/2) + C(n/2)=1+2C(n/2)$ for $n > 0$ because: \texttt{recurse(n)} will invoke \texttt{recurse( )} twice with input size $n/2$, and by definition, the \# times that \texttt{doPrimitive()} is executed when input = $n/2$ is: $C(n/2)$.
	\item Telescoping: \begin{align*}
		C(n)&=1+2C(n/2)\\
		&=1+2+4C(n/4)\\
		&=1+2^1+2^2+\cdots+2^k*C(n/2^k)
	\end{align*}
     We want $C(n/2^k)$ to eventually be $C(1)$. So, we have $1=n/2^k\implies n=2^k\implies k=\log{n}$.
     So, \begin{align*}
		C(n)&=1+2^1+2^2+\cdots+2^kC(1)\\
		&=1+2^1+2^2+\cdots+2^k\\
		2C(n)&=2^1+2^2+2^3+\cdots+2^{k+1}\\
		C(n)=2C(n)-C(n)&=2^{k+1}-1\\
		&=2^{\log{n}+1}-1\\&=2\cdot2^{\log{n}}-1\\
		&=2n-1
	\end{align*}
	\item So, $\O(n)$.
\end{itemize}

\end{document}